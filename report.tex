\documentclass{./llncs2e/llncs}
\usepackage{graphicx}
\usepackage{fixltx2e}
\usepackage{mathtools}
\usepackage[nolist,nohyperlinks]{acronym}
% Maintain images and tables within their respective sections
\usepackage[section]{placeins}

\usepackage{float}

% 
% Change the margins
% 
% \usepackage[margin=2.9cm]{geometry}

\begin{document}
\title{Your Thesis title}

\subtitle{Your Thesis subtitle}
\author{Pedro Alfaiate, pedro.alfaiate@tecnico.ulisboa.pt}
\institute{Instituto Superior Técnico}

\maketitle

%----------------------------------------------------
%NAVabstract
\begin{abstract}

\end{abstract}
%----------------------------------------------------
%NAVkeywords
\begin{keywords}

\end{keywords}
%----------------------------------------------------
%NAVintroduction
\section{Introduction (2/3pgs)}

%----------------------------------------------------
%NAVobjectives
\section{Objectives (1pg)}
The aim of this project is to provide architects -- or anyone interested in creating 3D models/geometry -- with an environment that fosters the Generative Design approach while being accessible wherever they are working.

%Why does it need to be accessible wherever the architect is?
Architects imagine buildings and will have to move their ideas to outside of their heads wherever they are; if they don't, they might forget them forever. We have to provide a means to translate ideas and it has to be accessible wherever they are.\textsuperscript{citation needed}

How can we provide a means to translate ideas? We make them -- the architects -- translate the idea into a program, we encourage them to use Generative Design.

Writing a program can degenerate into writing a sequence of atomic instructions if the author doesn't know how to simplify it \textsuperscript{citation needed}. We can't assume that architects are taught how to program\textsuperscript{citation needed}; it is our responsibility to encourage them to follow good programming practices and to make following those practices easy.

There are many well established tools that architects use to translate their ideas; if we're going to come up with a solution that can rival them than it must be as easy to try out as possible; it also has to provide a way to export results into those tools.

Summing up, the result of this project should:
\begin{enumerate}
	\item Be accessible wherever the architect is working; \label{obj:access}
	\item Encourage the Generative Design approach, which implies: \label{obj:gen-design}
	\begin{enumerate}
		\item Encouraging good programming practices; \label{obj:good-prog-practs} 
		\item Making writing programs easy; \label{obj:easy-program}
	\end{enumerate}
	\item Be as easy to try out as possible -- without a middle step to install it; \label{obj:no-install}
	\item Allow easy exportation of created 3D models to other modeling software -- easily integrating into the user's working process. \label{obj:inter-op}
\end{enumerate}


%----------------------------------------------------
%NAVrelatedwork
\section{Related Work (~17pgs)}
In this section we start by describing work done on making programming easy for everyone and extract the principles that should be followed to achieve it.

We move on to present some of the current widely used solutions -- or environments -- in the Generative Design community and evaluate their conformance with the principles defined earlier.

\subsection{Programming for Everyone}
Making programming easy for everyone has been an active area of research for many years. Some examples of programming languages designed with introducing people to programming were Logo\cite{papert1999logo} -- where the programmer explains the meaning of new words to a turtle -- and Smalltalk\cite{goldberg1983smalltalk} -- where a program the way a group of objects exchanges messages to solve a problem. Their authors are great defenders of the notion that learning how to solve/understand problems with a computer is a greater way of learning since it is a dynamic medium -- an example of that is the Dynabook\cite{Kay:2011:PCC:800193.1971922} a handheld computer prototype for children where programming was a first-class citizen.


%----------------------------------------------------
%NAVarchitecture
\section{Architecture (2/3pgs)}

%----------------------------------------------------
%NAVevaluation
\section{Evaluation (1/2pgs)}

%----------------------------------------------------
%NAVconclusions
\section{Conclusions}

%----------------------------------------------------
%NAVappendix
\newpage
\appendix
\section{Appendix}
\label{sec:attachments}

%----------------------------------------------------

% 
% Bibliography
% 
\bibliographystyle{plain} 

% replace example.bib with your .bib
\bibliography{report} 

\end{document}