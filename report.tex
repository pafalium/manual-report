\documentclass{./llncs2e/llncs}
\usepackage{graphicx}
\usepackage{fixltx2e}
\usepackage{mathtools}
\usepackage[nolist,nohyperlinks]{acronym}
% Maintain images and tables within their respective sections
\usepackage[section]{placeins}

\usepackage{float}

% 
% Change the margins
% 
% \usepackage[margin=2.9cm]{geometry}

\begin{document}
\title{Your Thesis title}

\subtitle{Your Thesis subtitle}
\author{Pedro Alfaiate, pedro.alfaiate@tecnico.ulisboa.pt}
\institute{Instituto Superior Técnico}

\maketitle

%----------------------------------------------------
%NAVabstract
\begin{abstract}

\end{abstract}
%----------------------------------------------------
%NAVkeywords
\begin{keywords}

\end{keywords}
%----------------------------------------------------
%NAVintroduction
\section{Introduction (2/3pgs)}

%----------------------------------------------------
%NAVobjectives
\section{Objectives (1pg)}
The aim of this project is to provide architects -- or anyone interested in creating 3D models/geometry -- with an environment that fosters the Generative Design approach while being accessible wherever they are working.

%Why does it need to be accessible wherever the architect is?
Architects imagine buildings and will have to move their ideas to outside of their heads wherever they are; if they don't, they might forget them never recalling them again. We have to provide a means to translate ideas and it has to be accessible wherever they are.\textsuperscript{citation needed}

How can we provide a means to translate ideas? We make them translate the idea into a program, we encourage them to use Generative Design.

Writing a program can degenerate into writing a sequence of atomic instructions if the author doesn't know how to simplify it \textsuperscript{citation needed}. We can't assume that architects are taught how to program\textsuperscript{citation needed}; it is our responsibility to encourage them to follow good programming practices and to make following those practices easy.

There are many well established tools that architects use to translate their ideas; if we're going to come up with a solution that can rival them than it must be as easy to try out as possible; and it also has to provide a way to export results into those tools.

This aim can be further decomposed into the following.
Summing up, the result of this project should:
\begin{enumerate}
	\item Be accessible wherever the architect is working; \label{obj:access}
	\item Encourage the Generative Design approach, which implies: label{obj:gen-design}
	\begin{enumerate}
		\item Encouraging good programming practices; \label{obj:good-prog-practs} 
		\item Making writing programs easy; \label{obj:easy-program}
	\end{enumerate}
	\item Be as easy to try out as possible -- i.e. without a middle step to install it; \label{obj:no-install}
	\item Allow easy exportation of created 3D models to other modeling software -- to be easily integrated with the user's working process. \label{obj:inter-op}
\end{enumerate}


%----------------------------------------------------
%NAVrelatedwork
\section{Related Work (~17pgs)}

%----------------------------------------------------
%NAVarchitecture
\section{Architecture (2/3pgs)}

%----------------------------------------------------
%NAVevaluation
\section{Evaluation (1/2pgs)}

%----------------------------------------------------
%NAVconclusions
\section{Conclusions}

%----------------------------------------------------
%NAVappendix
\newpage
\appendix
\section{Appendix}
\label{sec:attachments}

%----------------------------------------------------

% 
% Bibliography
% 
\bibliographystyle{plain} 

% replace example.bib with your .bib
\bibliography{report} 

\end{document}